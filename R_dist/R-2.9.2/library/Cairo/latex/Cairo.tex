\HeaderA{Cairo}{Create a new Cairo-based graphics device}{Cairo}
\aliasA{CairoJPEG}{Cairo}{CairoJPEG}
\aliasA{CairoPDF}{Cairo}{CairoPDF}
\aliasA{CairoPNG}{Cairo}{CairoPNG}
\aliasA{CairoPS}{Cairo}{CairoPS}
\aliasA{CairoSVG}{Cairo}{CairoSVG}
\aliasA{CairoTIFF}{Cairo}{CairoTIFF}
\aliasA{CairoWin}{Cairo}{CairoWin}
\aliasA{CairoX11}{Cairo}{CairoX11}
\keyword{device}{Cairo}
\begin{Description}\relax
\code{Cairo} initializes a new graphics device that uses the cairo
graphics library for rendering. The current implementation produces
high-quality PNG, JPEG, TIFF bitmap files, high resolution PDF files
with embedded fonts, SVG graphics and PostScript files. It also
provides X11 and Windows interactive graphics devices. Unlike other
devices it supports all graphics features including alpha blending,
anti-aliasing etc.

\code{CairoX11}, \code{CairoPNG}, \code{CairoPDF}, \code{CairoPS} and
\code{CairoSVG} are convenience wrappers of \code{Cairo} that take the
same arguments as the corresponding device it replaces such as
\code{X11}, \code{png}, \code{pdf}, etc. Use of the \code{Cairo}
function is encouraged as it is more flexible than the wrappers.
\end{Description}
\begin{Usage}
\begin{verbatim}
Cairo(width = 640, height = 480, file="", type="png", pointsize=12, 
      bg = "transparent", canvas = "white", units = "px", dpi = "auto",
      ...)

CairoX11(display=Sys.getenv("DISPLAY"), width = 7, height = 7,
         pointsize = 12, gamma = getOption("gamma"), bg = "transparent",
         canvas = "white", xpos = NA, ypos = NA, ...)
CairoPNG(filename = "Rplot%03d.png", width = 480, height = 480,
         pointsize = 12, bg = "white",  res = NA, ...)
CairoJPEG(filename = "Rplot%03d.jpeg", width = 480, height = 480,
         pointsize = 12, quality = 75, bg = "white", res = NA, ...)
CairoTIFF(filename = "Rplot%03d.tiff", width = 480, height = 480,
         pointsize = 12, bg = "white", res = NA, ...)
CairoPDF(file = ifelse(onefile, "Rplots.pdf","Rplot%03d.pdf"),
         width = 6, height = 6, onefile = TRUE, family = "Helvetica",
         title = "R Graphics Output", fonts = NULL, version = "1.1",
         paper = "special", encoding, bg, fg, pointsize, pagecentre)
CairoSVG(file = ifelse(onefile, "Rplots.svg", "Rplot%03d.svg"),
         width = 6, height = 6, onefile = TRUE, bg = "transparent",
         pointsize = 12, ...)
CairoWin(width = 7, height = 7, pointsize = 12,
         record = getOption("graphics.record"),
         rescale = c("R", "fit", "fixed"), xpinch, ypinch, bg =
         "transparent", canvas = "white", gamma = getOption("gamma"),
         xpos = NA, ypos = NA, buffered = getOption("windowsBuffered"),
         restoreConsole = FALSE, ...)
CairoPS(file = ifelse(onefile, "Rplots.ps", "Rplot%03d.ps"),
        onefile = TRUE, family, title = "R Graphics Output", fonts = NULL,
        encoding, bg, fg, width, height, horizontal, pointsize, paper,
        pagecentre, print.it, command, colormodel)
\end{verbatim}
\end{Usage}
\begin{Arguments}
\begin{ldescription}
\item[\code{width}] width of the plot area (also see \code{units}).
\item[\code{height}] height of the plot area (also see \code{units}).
\item[\code{file}] name of the file to be created or connection to write
to. Only PDF, PS and PNG types support connections. For X11
type \code{file} specifies the display name. If \code{NULL} or
\code{""} a reasonable default will be chosen which is
\code{"plot.type"} for file-oriented types and value of the
\code{DISPLAY} environment variable for X11. For image types
the file name can contain printf-style formatting expecting
one integer parameter which is the page number, such as
\code{"Rplot\%03d.png"}. The page numbers start at one.
\item[\code{type}] output type. This version of Cario supports "png", "jpeg"
and "tiff" bitmaps (png/tiff with transparent background), "pdf"
PDF-file with embedded fonts, "svg" SVG-file, "ps" PostScript-file,
"x11" X11 interactive window and "win" Windows graphics.
Depending on the support of various backends in cairo graphics some
of the options may not be available for your system. See
\code{\LinkA{Cairo.capabilities}{Cairo.capabilities}} function.
\item[\code{pointsize}] initial text size (in points).
\item[\code{canvas}] canvas color (must be opaque). The canvas is only used
by devices that display graphics on a screen and the canvas is only
visible only if bg is transparent.
\item[\code{bg}] plot background color (can include alpha-component or be
transparent alltogether).
\item[\code{units}] units for of the \code{width} and \code{height}
specifications. It can be any of \code{"px"} (pixels),
\code{"in"} (inches), \code{"pt"} (points), \code{"cm"}
(centimeters) or \code{"mm"} (millimeters).
\item[\code{dpi}] DPI used for the conversion of units to pixels. If set to
\code{"auto"} the DPI resolution will be determined by the
back-end.
\item[\code{...}] additional backend specific parameters (e.g. \code{quality}
setting for JPEG (0..100) and \code{compression} for TIFF
(0,1=none, 5=LZW (default), 7=JPEG, 8=Adobe Deflate))   

All parameters
listed below are defined by the other devices are are used by
the wrappers to make it easier replace other devices by
\code{Cairo}. They are described in detail in the documentation
corresponding to the device that is being replaced.
\item[\code{display}] X11 display, see \code{\LinkA{X11}{X11}}
\item[\code{gamma}] gamma correction
\item[\code{xpos}] see \code{\LinkA{X11}{X11}}
\item[\code{ypos}] see \code{\LinkA{X11}{X11}}
\item[\code{filename}] same as \code{file} in \code{Cairo}
\item[\code{res}] see \code{\LinkA{png}{png}}, will be mapped to \code{dpi}
for \code{Cairo}
\item[\code{quality}] quality of the jpeg, see \code{\LinkA{jpeg}{jpeg}}
\item[\code{onefile}] logical: if true (the default) allow multiple
figures in one file (see \code{\LinkA{pdf}{pdf}}). false is currently
not supported by vector devices
\item[\code{family}] font family, see \code{\LinkA{pdf}{pdf}}
\item[\code{title}] see \code{\LinkA{pdf}{pdf}} (ignored)
\item[\code{fonts}] see \code{\LinkA{pdf}{pdf}}, ignored, \code{Cairo}
automatically detects and embeds fonts
\item[\code{version}] PDF version, see \code{\LinkA{pdf}{pdf}} (ignored)
\item[\code{paper}] see \code{\LinkA{pdf}{pdf}} (ignored, \code{Cairo} uses device dimensions)
\item[\code{encoding}] see \code{\LinkA{pdf}{pdf}} (ignored, \code{Cairo} uses
native enconding except for symbols)
\item[\code{fg}] see \code{\LinkA{pdf}{pdf}} (ignored)
\item[\code{pagecentre}] see \code{\LinkA{pdf}{pdf}} (ignored, \code{Cairo}
uses device dimensions and thus it is irrelevant)
\item[\code{record}] see \code{\LinkA{windows}{windows}} (ignored)
\item[\code{rescale}] see \code{\LinkA{windows}{windows}} (ignored)
\item[\code{xpinch}] see \code{\LinkA{windows}{windows}} (ignored)
\item[\code{ypinch}] see \code{\LinkA{windows}{windows}} (ignored)
\item[\code{buffered}] see \code{\LinkA{windows}{windows}} (ignored, \code{Cairo}
always uses cache buffer)
\item[\code{restoreConsole}] see \code{\LinkA{windows}{windows}} (ignored)
\item[\code{horizontal}] see \code{\LinkA{postscript}{postscript}} (ignored)
\item[\code{print.it}] see \code{\LinkA{postscript}{postscript}} (ignored)
\item[\code{command}] see \code{\LinkA{postscript}{postscript}} (ignored)
\item[\code{colormodel}] see \code{\LinkA{postscript}{postscript}} (ignored,
\code{Cairo} always uses \code{RGB} or \code{ARGB})
\end{ldescription}
\end{Arguments}
\begin{Value}
The (invisible) return value is NULL if the device couldn't be created
or a \code{Cairo} object if successful. The vaule of the object is the
device number.
\end{Value}
\begin{Section}{Known issues}
\Itemize{
\item The X11 backend is quite slow. The reason is the cairographics
implementation of the backend, so we can't do much about
it. It should be possible to drop cairographics' Xlib
backend entirely and use image backend copied into an X11
window instead. We may try that in future releases.
\item TrueType (and OpenType) fonts are supported when this package is compiled
against a cairo graphics library configured with FreeType and
Fontconfig support. Therefore make sure have a cairo grpahics
library with all bell and whistles to get a good result.
\item R math symbols are supported, but require a TrueType "Symbol" font accessible
to Cairo under that name.
}
\end{Section}
\begin{SeeAlso}\relax
\code{\LinkA{CairoFonts}{CairoFonts}}
\end{SeeAlso}
\begin{Examples}
\begin{ExampleCode}
# very simple KDE
Cairo(600, 600, file="plot.png", type="png", bg="white")
plot(rnorm(4000),rnorm(4000),col="#ff000018",pch=19,cex=2) # semi-transparent red
dev.off() # creates a file "plot.png" with the above plot

# you can use any Cairo backend and get the same result
# vector, bitmap or on-screen
CairoPDF("plot.pdf", 6, 6, bg="transparent")
data(iris)
attach(iris)
plot(Petal.Length, rep(-0.03,length(Species)), xlim=c(1,7),
     ylim=c(0,1.7), xlab="Petal.Length", ylab="Density",
     pch=21, cex=1.5, col="#00000001", main = "Iris (yet again)",
     bg=c("#ff000020","#00ff0020","#0000ff20")[unclass(Species)])
for (i in 1:3)
  polygon(density(Petal.Length[unclass(Species)==i],bw=0.2),
    col=c("#ff000040","#00ff0040","#0000ff40")[i])
dev.off()
\end{ExampleCode}
\end{Examples}

