\HeaderA{ROCR.xval}{Data set: Artificial cross-validation data for use with ROCR}{ROCR.xval}
\keyword{datasets}{ROCR.xval}
\begin{Description}\relax
A mock data set containing 10 sets of predictions and
corresponding labels as would be obtained from 10-fold cross-validation.
\end{Description}
\begin{Usage}
\begin{verbatim}data(ROCR.xval)\end{verbatim}
\end{Usage}
\begin{Format}\relax
A two element list. The first element,
\code{ROCR.xval\$predictions}, is itself a 10 element list. Each of these 10
elements is a vector of numerical predictions for each cross-validation
run. Likewise, the second list entry, \code{ROCR.xval\$labels} is a 10
element list in which each element is a vector of true class
labels corresponding to the predictions.
\end{Format}
\begin{Examples}
\begin{ExampleCode}
# plot ROC curves for several cross-validation runs (dotted
# in grey), overlaid by the vertical average curve and boxplots
# showing the vertical spread around the average.
data(ROCR.xval)
pred <- prediction(ROCR.xval$predictions, ROCR.xval$labels)
perf <- performance(pred,"tpr","fpr")
plot(perf,col="grey82",lty=3)
plot(perf,lwd=3,avg="vertical",spread.estimate="boxplot",add=TRUE)
\end{ExampleCode}
\end{Examples}

